\documentclass{article}


% if you need to pass options to natbib, use, e.g.:
%     \PassOptionsToPackage{numbers, compress}{natbib}
% before loading neurips_2024


% ready for submission
\usepackage[final]{neurips_2024}


% to compile a preprint version, e.g., for submission to arXiv, add add the
% [preprint] option:
%     \usepackage[preprint]{neurips_2024}


% to compile a camera-ready version, add the [final] option, e.g.:
%     \usepackage[final]{neurips_2024}


% to avoid loading the natbib package, add option nonatbib:
%    \usepackage[nonatbib]{neurips_2024}


\usepackage[utf8]{inputenc} % allow utf-8 input
\usepackage[T1]{fontenc}    % use 8-bit T1 fonts
\usepackage{hyperref}       % hyperlinks
\usepackage{url}            % simple URL typesetting
\usepackage{booktabs}       % professional-quality tables
\usepackage{amsfonts}       % blackboard math symbols
\usepackage{amsmath}        % equations
\usepackage{nicefrac}       % compact symbols for 1/2, etc.
\usepackage{microtype}      % microtypography
\usepackage{xcolor}         % colors
\usepackage{graphicx}       % adding graphics


\title{ECE C147/C247 Report: Predicting Keystrokes from Electromyography Signals}


% The \author macro works with any number of authors. There are two commands
% used to separate the names and addresses of multiple authors: \And and \AND.
%
% Using \And between authors leaves it to LaTeX to determine where to break the
% lines. Using \AND forces a line break at that point. So, if LaTeX puts 3 of 4
% authors names on the first line, and the last on the second line, try using
% \AND instead of \And before the third author name.


\author{
  Aparna Hariharan\\
  UID: 705715914 \\
  Department of Computer Science\\
  University of California, Los Angeles\\
  \texttt{aparnahariharan@ucla.edu} \\
    \And
    Joseph Janssen \\
    UID: 905797640 \\
  Department of Computer Science\\
  University of California, Los Angeles\\
  \texttt{
josephjanssen11@gmail.com} \\
\And
Arathi Nair\\
    UID: 805749584\\
  Department of Computer Science\\
  University of California, Los Angeles\\
  \texttt{
arathinair@g.ucla.edu} \\
\And
Sruthi Rangarajan\\
UID: 405720436 \\
  Department of Computer Science\\
  University of California, Los Angeles\\
  \texttt{
sruthi.ranga3@gmail.com} \\
  % examples of more authors
  % \And
  % Coauthor \\
  % Affiliation \\
  % Address \\
  % \texttt{email} \\
  % \And
  % Coauthor \\
  % Affiliation \\
  % Address \\
  % \texttt{email} \\
  % \And
  % Coauthor \\
  % Affiliation \\
  % Address \\
  % \texttt{email} \\
  % \And
  % Coauthor \\
  % Affiliation \\
  % Address \\
  % \texttt{email} \\
}


\begin{document}

\maketitle

\begin{abstract}
  Surface electromyography (sEMG) signals have the potential to enable intuitive and efficient human-computer interaction, particularly for text entry applications. In this paper, we preprocess the emg2qwerty dataset using min-max normalization, time stretching, and downsampling to enhance signal quality. We also evaluate multiple deep learning architectures, including the baseline CNN model, after incorporating different regularization techniques like dropout, batch normalization, L2 regularization, and elastic net. We contrasted the baseline convolutional model with an alternative recurrent neural network (RNN) model and a hybrid model of CNN and RNN. We found that the best performing model was achieved through adding L2 regularization and a dropout layer to the baseline CNN architecture. Additionally, decreasing the sampling rate increased the resulting CER. These findings highlight the role that techniques such as data preprocessing and architectural choices can have in model performance while maintaining that not all of these techniques always result in performance improvements.
  
  % The abstract paragraph should be indented \nicefrac{1}{2}~inch (3~picas) on
  % both the left- and right-hand margins. Use 10~point type, with a vertical
  % spacing (leading) of 11~points.  The word \textbf{Abstract} must be centered,
  % bold, and in point size 12. Two line spaces precede the abstract. The abstract
  % must be limited to one paragraph.
\end{abstract}


\section{Introduction}
Accurately decoding surface electromyography (sEMG) signals for text entry remains a challenging problem in human-computer interaction. The emg2qwerty dataset provides a rich source of high-resolution sEMG recordings, capturing fine motor control dynamics. Given the complexity and variability of sEMG signals, improving model generalization is crucial for robust performance across different contexts. Extensive data preprocessing techniques, including downsampling, were explored, but they worsened performance. This suggests that the dataset's inherent diversity—even within a single subject—already introduces sufficient variability for effective learning. Consequently, the focus shifted toward model-based generalization techniques, including dropout, batch normalization, L2 regularization, and elastic net regularization.

In addition to these generalization techniques, a recurrent neural network (RNN) was investigated as an alternative to the given convolutional neural network (CNN). Unlike feedforward architectures, RNNs are well-suited for capturing temporal dependencies in sequential data. Since sEMG signals exhibit strong temporal correlations due to the continuous nature of muscle activations, an RNN has the potential to better model these dependencies and improve decoding accuracy. A hybrid model with both RNN and CNN components was also explored. This report details the approach to preprocessing, model selection, and evaluation, with the goal of optimizing character error rate (CER) while ensuring strong generalization performance.

\section{Methods}
We chose to run a majority of our tests using 30 epochs, including the baseline model provided to us. During experimentation for techniques that approached the baseline validation CER, we chose to run those tests for a higher number of epochs, which are specified in the results sections.
\subsection{Data Preprocessing}
To improve the generalization of the model and enhance the quality of input features, several data preprocessing techniques were applied to the sEMG signals. These methods aimed to reduce noise, normalize signal values, and introduce variations that could improve model robustness.
Two different preprocessing trials were conducted: one using only downsampling and another incorporating all three augmentation techniques. Both of these trials included the baseline preprocessing pipeline.

\subsubsection{Downsampling}
To reduce the computational load and eliminate high-frequency noise, the raw sEMG signals were downsampled by a specified factor, \textit{n}. This process involved selecting every \textit{n}-th sample from the original signal while preserving temporal dependencies. Downsampling helps mitigate redundancy in high-resolution data while maintaining critical signal information.
\subsubsection{Min-Max Normalization}
Min-max normalization was applied to scale the sEMG signals within a defined range, in this case [0,1]. This technique ensures that all input features have comparable magnitudes, improving numerical stability during training. Given a signal $x$, the transformation follows:
\begin{equation}
x' = \frac{x - x_{\min}}{x_{\max} - x_{\min}}
\end{equation}
where $x_{\min}$ and $x_{\max}$ represent the minimum and maximum values of the signal, respectively.
\subsubsection{Time Stretching}
Time stretching was employed to introduce variations in the temporal dynamics of sEMG signals. This augmentation technique expands or compresses the x-axis (time) of the signal without altering its frequency content. It aids in improving model robustness by exposing it to diverse time-scaled versions of the data while preserving the frequency patterns.
\subsection{Regularization Techniques}
To enhance the generalization capability of our model and mitigate overfitting, several regularization techniques were incorporated into the training pipeline. These methods aim to improve model robustness by restricting complexity and preventing the network from overfitting to the variability of the emg2qwerty training dataset. By integrating these regularization techniques, our implementation aimed to develop a model that generalizes well across different datasets.
\subsubsection{Dropout}
Dropout is a stochastic regularization technique that randomly deactivates a subset of neurons during each training iteration. This prevents the network from relying too heavily on specific neurons and encourages redundancy in feature learning, which improves generalization to unseen data. In our implementation, dropout was applied, with probability values chosen through empirical validation.
\subsubsection{Batch Normalization}
Batch normalization (BN) standardizes the activations within each mini-batch by normalizing them to have unit statistics, followed by a learnable affine transformation. Learning thus becomes simpler because parameters in the lower layers do not change the statistics of the input to a given layer. BN was applied to ensure consistent feature scaling, allowing the network to adapt effectively to variations in the input signal.
\subsubsection{L2 Regularization}
L2 regularization, or weight decay, penalizes large weight magnitudes by adding a term proportional to the squared L2 norm of the weights to the loss function. This encourages the model to maintain parameters closer to 0, thereby reducing overfitting. The loss function is as follows:
\begin{equation}
\mathcal{L}_{\text{L2}} = \mathcal{L} + \lambda \sum_{i} w_i^2
\end{equation}
where:
\begin{itemize}
    \item \( \mathcal{L}_{\text{L2}} \) is the total loss with L2 regularization.
    \item \( \mathcal{L} \) is the original loss function (e.g., cross-entropy or mean squared error).
    \item \( w_i \) represents the weight parameters of the model.
    \item \( \lambda \) is the L2 regularization strength (hyperparameter).
\end{itemize}
\subsubsection{Elastic Net Regularization}
Elastic net combines L1 and L2 regularization, promoting both sparsity and weight decay.This hybrid approach is especially useful in high-dimensional feature spaces, as it leverages L1 regularization to promote sparse feature selection while using L2 to prevent excessive sparsity and improve stability.
Given the complex structure of sEMG data, Elastic Net was explored as an alternative to pure L2 regularization to assess its impact on feature selection and model generalization. The loss function is as follows:
\begin{equation}
\mathcal{L}_{\text{ElasticNet}} = \mathcal{L} + \lambda_1 \sum_{i} |w_i| + \lambda_2 \sum_{i} w_i^2
\end{equation}
where:
\begin{itemize}
    \item \( \mathcal{L}_{\text{ElasticNet}} \) is the total loss with Elastic Net regularization.
    \item \( \mathcal{L} \) is the original loss function.
    \item \( w_i \) represents the weight parameters of the model.
    \item \( \lambda_1 \) controls the strength of L1 regularization, encouraging sparsity.
    \item \( \lambda_2 \) controls the strength of L2 regularization, promoting weight decay.
\end{itemize}
\subsection{Model Architectures}
To effectively decode sEMG signals for text entry, multiple neural network architectures were explored, each designed to capture different aspects of the input data. This section describes the convolutional neural network (CNN), the recurrent neural network (RNN), and a hybrid approach combining both architectures. By evaluating these architectures, we assessed their effectiveness in improving character error rate (CER).
\subsubsection{Convolutional Neural Network (CNN)}
A convolutional neural network (CNN) was employed as the baseline architecture due to its ability to extract spatial patterns from high-dimensional sEMG data. The CNN follows a time-depth separable convolutional encoder design. The convolutional layers process the spectrogram-normalized input features, applying temporal convolutions. Despite CNNs' strong ability to extract spatial features, their inability to model long-term temporal dependencies prompted further exploration of alternative architectures.
\subsubsection{Recurrent Neural Network (RNN)}
Given the sequential nature of sEMG signals, an RNN-based architecture was investigated as an alternative to the CNN. Unlike CNNs, RNNs leverage recurrent connections to capture long-range dependencies, making them well-suited for modeling time-series data. The RNN model employs a stacked LSTM or GRU architecture, with configurable parameters such as hidden size, number of layers, and bidirectionality. These recurrent layers process the extracted features, maintaining temporal dependencies across time steps. 
\subsubsection{Hybrid (CNN + RNN)}
To combine the strengths of both architectures, a hybrid CNN-RNN model was developed. In this approach, convolutional layers first extract spatial features from sEMG signals, which are then passed to recurrent layers to capture temporal dependencies. It uses both a convolutional encoder to perform local temporal modeling and a recurrent encoder (LSTM or GRU) to account for long-range dependencies. 

\section{Results}
\begin{table}[h]
  \caption{Comparison of different models and techniques with their Validation Character Error Rate (CER) and Validation Loss. The model with the lowest validation CER is bolded.}
  \label{tab:model_cer_loss}
  \centering
  \begin{tabular}{lcc}
    \toprule
    \textbf{Model + Technique} & \textbf{Validation CER} & \textbf{Validation Loss} \\
    \midrule
    Baseline Model and Data & 26.1631 & 0.8681 \\
    Baseline Model + Downsampling & 93.9965 & 3.0466 \\
    Baseline Model + Downsampling, \\ \hspace{0.5cm} Time Stretching, \& Min-Max Norm. & 100.0000 & 3.6808 \\
    Baseline Model + Dropout=0.2 & 29.3531 & 0.9317 \\
    Baseline Model + Dropout=0.3 & 29.6854 & 0.9425 \\
    Baseline Model + Dropout=0.4 & 30.8595 & 0.9559 \\
    Baseline Model + Batch Norm & 82.2330 & 12.1359 \\
    Baseline Model + L2 Regularization & 21.5773 & 1.5503 \\
    \textbf{Baseline Model + L2 Regularization + Dropout=0.2} & \textbf{20.7798} & \textbf{1.5496} \\
    Baseline Model + Elastic Net + Dropout=0.2 & 22.3748 & 2.2133 \\
    RNN Encoder Model with 2 Layers \& 256 Nodes & 49.2025 & 1.3598 \\
    RNN Encoder Model with 3 Layers \& 256 Nodes & 73.4603 & 3.0090 \\
    RNN Encoder Model with 3 Layers \& 512 Nodes & 73.8148 & 2.6043 \\
    RNN + CNN Hybrid Model with 3 Layers \& 256 Nodes \\ \hspace{0.5cm} + Downsampling and Min-Max Norm. & 100.0000 & 3.2801 \\
    
    \bottomrule
  \end{tabular}
\end{table}

Some key plots are shown in the current section, but more can be found in the Appendix~\ref{sec:appendix}.

\subsection{Baseline}
\begin{figure}[ht]
    \centering
    \begin{minipage}{0.48\textwidth}
        \centering
        \includegraphics[width=\linewidth]{Baseline_CER.png}
        \caption{Training and validation CERs for baseline CNN model across 30 epochs.}
    \end{minipage}
    \hfill
    \begin{minipage}{0.48\textwidth}
        \centering
        \includegraphics[width=\linewidth]{Baseline_loss.png}
        \caption{Training and validation losses for baseline CNN model across 30 epochs.}
    \end{minipage}
\end{figure}
As shown in in {Table~\ref{tab:model_cer_loss}}, running the baseline model resulted in a validation CER of 26.1631. 
\subsection{Preprocessing the Data by Downsampling}
After downsampling the data, the resulting validation CER was 93.9965. This is drastically higher than the validation CER attained by simply just running the baseline model on the data. The intention of downsampling is to remove high-frequency noise through reducing the amount of data, but the results in {Table~\ref{tab:model_cer_loss}} show that this reduced data was insufficient for the model to make accurate predictions.
\subsection{Preprocessing the Data by Downsampling, Min-Max Normalization, and Time Stretching}
As shown in the results in {Table~\ref{tab:model_cer_loss}}, this combination of data augmentation techniques performed extremely poorly with the validation CER being 100.0000. Min-max normalization and time stretching were meant to scale the data into a consistent range and modify the speed or duration of the signal without changing its overall shape.  However, this combination of data augmentation techniques likely stripped essential information needed for accurate predictions, causing the increase in the validation CER. 
\subsection{Adding a Dropout Layer}
As shown in {Table~\ref{tab:model_cer_loss}}, we experimented with incorporating a dropout layer at three different probabilities: 0.2, 0.3, and 0.4. The results indicate that the validation CER values for all three configurations were consistently slightly higher than the validation CER of the baseline model. This suggests that while Dropout introduces regularization by preventing overfitting, it may also lead to a slight degradation in performance when applied independently.

\subsection{Adding L2 Regularization}
We added L2 Regularization to the baseline CNN model with $\lambda$ = 0.0001. After 52 epochs, this technique resulted in a validation CER of 21.5773, the second-lowest validation CER recorded amongst all of the techniques we tried, as seen in {Table~\ref{tab:model_cer_loss}}. This technique proved to be the first that improved Validation CER below that of the baseline model.
\subsection{Combining L2 Regularization and a Dropout Layer}
Our results, as shown in {Table~\ref{tab:model_cer_loss}}, indicate that the combination of a dropout layer (probability = 0.2) and L2 regularization ($\lambda$ = 0.0001) yielded the best performance among all tested configurations, with a validation CER of 20.7798 after 50 epochs. 
\subsection{Combining Elastic Net Regularization and a Dropout Layer}
Our results indicate that the combination of dropout and elastic net performed better than the baseline validation CER, yet not as well as the combination of dropout and L2 regularization. We ran this experiment with $\lambda_{1}$ = 0.00001 and $\lambda_2$ = 0.0001 values. The validation CER for elastic net and a dropout layer (probability = 0.2) is 22.3748 after 50 epochs, as shown in {Table~\ref{tab:model_cer_loss}}.
\subsection{Adding Batch Normalization}
When adding 1D batch normalization and running for 43 epochs, the results in {Table~\ref{tab:model_cer_loss}} show that the validation CER is higher than the baseline model at 82.2330. Further iterations of 1D batch normalization cause the validation CER to increase back up to 100.00 after hitting the minimum value of 82.2330. 
\subsection{RNN}
After trying multiple modification methods on the baseline CNN model, we implemented a RNN Encoder Model and tested it with a variation in the number of layers and input nodes. As per the results shown in {Table~\ref{tab:model_cer_loss}}, running the RNN Encoder Model with 2 Layers and 256 Nodes results in a validation CER of 49.2025. When the number of layers is increased, running the RNN Encoder Model with 3 Layers and 256 Nodes results in a validation CER of 73.4603 after 40 epochs. Furthermore, when the number of nodes is increased,  running the RNN Encoder Model with 3 Layers and 512 Nodes results in a validation CER of 73.8148, a CER very similar to the previous configuration of 3 Layers and 256 Nodes. This leads us to hypothesize that the increase in layers is more detrimental to the robustness of the model than the increase in nodes. Overall, a Recurrent Neural Network Encoder Model proved to perform more poorly than the CNN model with our architectural modifications. 
\subsection{RNN+CNN}
To further our experiments with the RNN Encoder Model, we created a hybrid RNN+CNN model that we ran with 3 layers and 256 nodes alongside augmentation of the data using downsampling and min-max normalization. The results in {Table~\ref{tab:model_cer_loss}} show that this combination technique resulted in a validation CER of 100.0000.

\section{Discussion}

\subsection{Effectiveness of Dropout and L2 Regularization}
We hypothesized that the combination of dropout and L2 regularization would outperform other configurations because they address two different aspects of overfitting. Dropout reduces dependencies between neurons, while L2 regularization constrains model complexity by discouraging extreme weight values. Our results show that this combination outperforms using L2 regularization alone with the baseline model. This is because dropout forces the model to utilize multiple neurons for the same tasks, resulting in greater robustness and generalization compared to relying solely on L2 regularization.

This balance between dropout and L2 regularization likely enhances the model’s ability to generalize to unseen data, reducing overfitting without sacrificing performance. Consequently, the validation CER is lower in models that incorporate both techniques rather than using them in isolation.

\subsection{Challenges with Batch Normalization}
In contrast, our results indicate that incorporating batch normalization in the baseline model leads to poor performance. During early epochs, batch normalization causes a slight decrease in the CER, but as training continues, the CER begins to rise again and plateaus at 100. This pattern may be due to discrepancies between mini-batch statistics and the running averages accumulated during training, which can negatively impact model stability. This could also be attributed to persistent issues faced in the evaluation pipeline, where both training and validation CERs plateaued at 100. 

Despite this issue, when training and validation loss were examined (Figure~\ref{fig:Batchnorm_loss}), a general downward trend was observed, indicating that the model continued optimizing its objective. Also, running the batch normalization variant of the model caused the notebook to time out at around 40 epochs multiple times, preventing a complete evaluation of its overall performance.

\subsection{Impact of Preprocessing Techniques}
Our results suggest that not all preprocessing techniques are beneficial, and in some cases, they can significantly reduce performance. For instance, while downsampling is often used to reduce computation time, it resulted in a validation CER of 93.99, indicating that the model struggled to differentiate between keystrokes.

Moreover, combining downsampling with time stretching and min-max normalization caused the model’s performance to collapse entirely. We initially hypothesized that this combination would help the model learn features robustly; however, our findings suggest that while normalization can be useful in certain contexts, it must be carefully tuned to avoid distorting the original signal. This could also be attributed to persistent issues faced in the evaluation pipeline, where both training and validation CERs plateaued at 100.
\subsection{Limitations of RNN-Based Architectures}
We also observed that RNN-based models did not perform as well as expected. The RNN encoder with three layers and 256 nodes achieved a validation CER of 73.46, which was significantly worse than the best-performing CNN model. This poor performance may be due to vanishing gradients or difficulty in capturing long-term dependencies in the sEMG signals.

While RNNs are typically effective for sequential data, our results indicate that additional modifications, such as layer reordering or alternative optimization techniques, may be necessary to improve performance. Additionally, increasing the number of layers from 2 to 3 and the number of nodes from 256 to 512 led to even worse performance than our original RNN, likely due to RNNs being prone to overfitting. The hybrid CNN-RNN architecture performed poorly, with a plateaued validation CER of 100, possibly due to issues in the evaluation pipeline.

\subsection{Key Takeaways}
Ultimately, our findings highlight several key insights:
\begin{itemize}
    \item Regularization techniques such as dropout and L2 regularization play a crucial role in preventing overfitting, leading to better generalization.
    \item Preprocessing techniques must be carefully selected, as excessive transformations and noise can strip away critical information and degrade performance.
    \item RNN-based architectures have potential but require additional refinements from the configurations we tested. The poor performance in our implementation was likely due to external factors, such as the evaluation pipeline, rather than inherent flaws in the model itself.
\end{itemize}


\section{References}
\bibitem{hannun2017sequence} 
Hannun, A. (2017). Sequence modeling with CTC. \textit{Distill}, \textit{2}(11), e8. 
\url{https://doi.org/10.23915/distill.00008}.

\bibitem{sivakumar2024emg2qwerty} 
Sivakumar, V., Seely, J., Du, A., Bittner, S. R., Berenzweig, A., Bolarinwa, A., Gramfort, A., \& Mandel, M. I. (2024).  
\textit{emg2qwerty: A Large Dataset with Baselines for Touch Typing using Surface Electromyography}.  
arXiv preprint arXiv:2410.20081.  
Retrieved from \url{https://arxiv.org/abs/2410.20081}.

\section{Appendix}
\label{sec:appendix}
\subsection{Additional Figures}
\begin{figure}[ht]
  \centering
  \includegraphics[width=0.8\linewidth]{Dropout0.4_CER.png} 
  \caption{Baseline CNN Model CER with Dropout=0.4.}
  \label{fig:dropout0.4_cer}
\end{figure}

\begin{figure}[ht]
  \centering
  \includegraphics[width=0.8\linewidth]{Dropout0.3_CER.png} 
  \caption{Baseline CNN Model CER with Dropout=0.3.}
  \label{fig:dropout0.3_cer}
\end{figure}

\begin{figure}[ht]
  \centering
  \includegraphics[width=0.8\linewidth]{Dropout0.2_CER.png} 
  \caption{Baseline CNN Model CER with Dropout=0.2.}
  \label{fig:dropout0.2_cer}
\end{figure}
\begin{figure}[ht]
  \centering
  \includegraphics[width=0.8\linewidth]{L2Reg_CER.png} 
  \caption{Baseline CNN Model CER with L2 Regularization.}
  \label{fig:L2_cer}
\end{figure}

\begin{figure}[ht]
  \centering
  \includegraphics[width=0.8\linewidth]{L2+Dropout0.2_CER.png} 
  \caption{Baseline CNN Model CER with Dropout=0.2 and L2 Regularization.}
  \label{fig:dropout0.2+L2_cer}
\end{figure}
\begin{figure}[ht]
      \centering
      \includegraphics[width=0.8\linewidth]{Styles/BatchNorm_CER.png} 
      \caption{Baseline CNN Model CER with Batch Normalization.}
      \label{fig:Batchnorm_cer}
\end{figure}
\begin{figure}[ht]
      \centering
      \includegraphics[width=0.8\linewidth]{Styles/BatchNorm_loss.png} 
      \caption{Baseline CNN Model Training/Validation Loss with Batch Normalization.}
      \label{fig:Batchnorm_loss}
\end{figure}

\begin{figure}[ht]
  \centering
  \includegraphics[width=0.8\linewidth]{ElasticNet+Dropout0.2_CER.png} 
  \caption{Baseline CNN Model CER with Dropout=0.2 and Elastic Net.}
  \label{fig:dropout0.2+elasticnet_cer}
\end{figure}

\begin{figure}[ht]
  \centering
  \includegraphics[width=0.8\linewidth]{RNN_CER.png} 
  \caption{RNN Encoder Model with 2 Layers \& 256 Nodes CER.}
  \label{fig:RNN_cer}
\end{figure}

\begin{figure}[ht]
  \centering
  \includegraphics[width=0.8\linewidth]{RNN_3_256_CER.png} 
  \caption{RNN Encoder Model with 3 Layers \& 256 Nodes CER.}
  \label{fig:RNN_3_256_cer}
\end{figure}

\begin{figure}[ht]
  \centering
  \includegraphics[width=0.8\linewidth]{RNN_3_512_CER.png} 
  \caption{RNN Encoder Model with 3 Layers \& 512 Nodes CER.}
  \label{fig:RNN_3_512_cer}
\end{figure}
    
\begin{figure}[ht]

  \centering
  \includegraphics[width=0.8\linewidth]{Hybrid_3_256_CER.png} 
  \caption{RNN + CNN Hybrid Model with 3 Layers \& 256 Nodes CER.}
  \label{fig:Hybrid_3_256_cer}

\end{figure}

\begin{figure}[ht]

  \centering
  \includegraphics[width=0.8\linewidth]{Hybrid_3_256_loss.png} 
  \caption{RNN + CNN Hybrid Model with 3 Layers \& 256 Nodes loss.}
  \label{fig:Hybrid_3_256_loss}

\end{figure}


\end{document}
% this is the format below

% \section{Submission of papers to NeurIPS 2024}


% Please read the instructions below carefully and follow them faithfully.


% \subsection{Style}


% Papers to be submitted to NeurIPS 2024 must be prepared according to the
% instructions presented here. Papers may only be up to {\bf nine} pages long,
% including figures. Additional pages \emph{containing only acknowledgments and
% references} are allowed. Papers that exceed the page limit will not be
% reviewed, or in any other way considered for presentation at the conference.


% The margins in 2024 are the same as those in previous years.


% Authors are required to use the NeurIPS \LaTeX{} style files obtainable at the
% NeurIPS website as indicated below. Please make sure you use the current files
% and not previous versions. Tweaking the style files may be grounds for
% rejection.


% \subsection{Retrieval of style files}


% The style files for NeurIPS and other conference information are available on
% the website at
% \begin{center}
%   \url{http://www.neurips.cc/}
% \end{center}
% The file \verb+neurips_2024.pdf+ contains these instructions and illustrates the
% various formatting requirements your NeurIPS paper must satisfy.


% The only supported style file for NeurIPS 2024 is \verb+neurips_2024.sty+,
% rewritten for \LaTeXe{}.  \textbf{Previous style files for \LaTeX{} 2.09,
%   Microsoft Word, and RTF are no longer supported!}


% The \LaTeX{} style file contains three optional arguments: \verb+final+, which
% creates a camera-ready copy, \verb+preprint+, which creates a preprint for
% submission to, e.g., arXiv, and \verb+nonatbib+, which will not load the
% \verb+natbib+ package for you in case of package clash.


% \paragraph{Preprint option}
% If you wish to post a preprint of your work online, e.g., on arXiv, using the
% NeurIPS style, please use the \verb+preprint+ option. This will create a
% nonanonymized version of your work with the text ``Preprint. Work in progress.''
% in the footer. This version may be distributed as you see fit, as long as you do not say which conference it was submitted to. Please \textbf{do
%   not} use the \verb+final+ option, which should \textbf{only} be used for
% papers accepted to NeurIPS.


% At submission time, please omit the \verb+final+ and \verb+preprint+
% options. This will anonymize your submission and add line numbers to aid
% review. Please do \emph{not} refer to these line numbers in your paper as they
% will be removed during generation of camera-ready copies.


% The file \verb+neurips_2024.tex+ may be used as a ``shell'' for writing your
% paper. All you have to do is replace the author, title, abstract, and text of
% the paper with your own.


% The formatting instructions contained in these style files are summarized in
% Sections \ref{gen_inst}, \ref{headings}, and \ref{others} below.


% \section{General formatting instructions}
% \label{gen_inst}


% The text must be confined within a rectangle 5.5~inches (33~picas) wide and
% 9~inches (54~picas) long. The left margin is 1.5~inch (9~picas).  Use 10~point
% type with a vertical spacing (leading) of 11~points.  Times New Roman is the
% preferred typeface throughout, and will be selected for you by default.
% Paragraphs are separated by \nicefrac{1}{2}~line space (5.5 points), with no
% indentation.


% The paper title should be 17~point, initial caps/lower case, bold, centered
% between two horizontal rules. The top rule should be 4~points thick and the
% bottom rule should be 1~point thick. Allow \nicefrac{1}{4}~inch space above and
% below the title to rules. All pages should start at 1~inch (6~picas) from the
% top of the page.


% For the final version, authors' names are set in boldface, and each name is
% centered above the corresponding address. The lead author's name is to be listed
% first (left-most), and the co-authors' names (if different address) are set to
% follow. If there is only one co-author, list both author and co-author side by
% side.


% Please pay special attention to the instructions in Section \ref{others}
% regarding figures, tables, acknowledgments, and references.


% \section{Headings: first level}
% \label{headings}


% All headings should be lower case (except for first word and proper nouns),
% flush left, and bold.


% First-level headings should be in 12-point type.


% \subsection{Headings: second level}


% Second-level headings should be in 10-point type.


% \subsubsection{Headings: third level}


% Third-level headings should be in 10-point type.


% \paragraph{Paragraphs}


% There is also a \verb+\paragraph+ command available, which sets the heading in
% bold, flush left, and inline with the text, with the heading followed by 1\,em
% of space.


% \section{Citations, figures, tables, references}
% \label{others}


% These instructions apply to everyone.


% \subsection{Citations within the text}


% The \verb+natbib+ package will be loaded for you by default.  Citations may be
% author/year or numeric, as long as you maintain internal consistency.  As to the
% format of the references themselves, any style is acceptable as long as it is
% used consistently.


% The documentation for \verb+natbib+ may be found at
% \begin{center}
%   \url{http://mirrors.ctan.org/macros/latex/contrib/natbib/natnotes.pdf}
% \end{center}
% Of note is the command \verb+\citet+, which produces citations appropriate for
% use in inline text.  For example,
% \begin{verbatim}
%    \citet{hasselmo} investigated\dots
% \end{verbatim}
% produces
% \begin{quote}
%   Hasselmo, et al.\ (1995) investigated\dots
% \end{quote}


% If you wish to load the \verb+natbib+ package with options, you may add the
% following before loading the \verb+neurips_2024+ package:
% \begin{verbatim}
%    \PassOptionsToPackage{options}{natbib}
% \end{verbatim}


% If \verb+natbib+ clashes with another package you load, you can add the optional
% argument \verb+nonatbib+ when loading the style file:
% \begin{verbatim}
%    \usepackage[nonatbib]{neurips_2024}
% \end{verbatim}


% As submission is double blind, refer to your own published work in the third
% person. That is, use ``In the previous work of Jones et al.\ [4],'' not ``In our
% previous work [4].'' If you cite your other papers that are not widely available
% (e.g., a journal paper under review), use anonymous author names in the
% citation, e.g., an author of the form ``A.\ Anonymous'' and include a copy of the anonymized paper in the supplementary material.


% \subsection{Footnotes}


% Footnotes should be used sparingly.  If you do require a footnote, indicate
% footnotes with a number\footnote{Sample of the first footnote.} in the
% text. Place the footnotes at the bottom of the page on which they appear.
% Precede the footnote with a horizontal rule of 2~inches (12~picas).


% Note that footnotes are properly typeset \emph{after} punctuation
% marks.\footnote{As in this example.}


% \subsection{Figures}


% \begin{figure}
%   \centering
%   \fbox{\rule[-.5cm]{0cm}{4cm} \rule[-.5cm]{4cm}{0cm}}
%   \caption{Sample figure caption.}
% \end{figure}


% All artwork must be neat, clean, and legible. Lines should be dark enough for
% purposes of reproduction. The figure number and caption always appear after the
% figure. Place one line space before the figure caption and one line space after
% the figure. The figure caption should be lower case (except for first word and
% proper nouns); figures are numbered consecutively.


% You may use color figures.  However, it is best for the figure captions and the
% paper body to be legible if the paper is printed in either black/white or in
% color.


% \subsection{Tables}


% All tables must be centered, neat, clean and legible.  The table number and
% title always appear before the table.  See Table~\ref{sample-table}.


% Place one line space before the table title, one line space after the
% table title, and one line space after the table. The table title must
% be lower case (except for first word and proper nouns); tables are
% numbered consecutively.


% Note that publication-quality tables \emph{do not contain vertical rules.} We
% strongly suggest the use of the \verb+booktabs+ package, which allows for
% typesetting high-quality, professional tables:
% \begin{center}
%   \url{https://www.ctan.org/pkg/booktabs}
% \end{center}
% This package was used to typeset Table~\ref{sample-table}.


% \begin{table}
%   \caption{Sample table title}
%   \label{sample-table}
%   \centering
%   \begin{tabular}{lll}
%     \toprule
%     \multicolumn{2}{c}{Part}                   \\
%     \cmidrule(r){1-2}
%     Name     & Description     & Size ($\mu$m) \\
%     \midrule
%     Dendrite & Input terminal  & $\sim$100     \\
%     Axon     & Output terminal & $\sim$10      \\
%     Soma     & Cell body       & up to $10^6$  \\
%     \bottomrule
%   \end{tabular}
% \end{table}

% \subsection{Math}
% Note that display math in bare TeX commands will not create correct line numbers for submission. Please use LaTeX (or AMSTeX) commands for unnumbered display math. (You really shouldn't be using \$\$ anyway; see \url{https://tex.stackexchange.com/questions/503/why-is-preferable-to} and \url{https://tex.stackexchange.com/questions/40492/what-are-the-differences-between-align-equation-and-displaymath} for more information.)

% \subsection{Final instructions}

% Do not change any aspects of the formatting parameters in the style files.  In
% particular, do not modify the width or length of the rectangle the text should
% fit into, and do not change font sizes (except perhaps in the
% \textbf{References} section; see below). Please note that pages should be
% numbered.


% \section{Preparing PDF files}


% Please prepare submission files with paper size ``US Letter,'' and not, for
% example, ``A4.''


% Fonts were the main cause of problems in the past years. Your PDF file must only
% contain Type 1 or Embedded TrueType fonts. Here are a few instructions to
% achieve this.


% \begin{itemize}


% \item You should directly generate PDF files using \verb+pdflatex+.


% \item You can check which fonts a PDF files uses.  In Acrobat Reader, select the
%   menu Files$>$Document Properties$>$Fonts and select Show All Fonts. You can
%   also use the program \verb+pdffonts+ which comes with \verb+xpdf+ and is
%   available out-of-the-box on most Linux machines.


% \item \verb+xfig+ "patterned" shapes are implemented with bitmap fonts.  Use
%   "solid" shapes instead.


% \item The \verb+\bbold+ package almost always uses bitmap fonts.  You should use
%   the equivalent AMS Fonts:
% \begin{verbatim}
%    \usepackage{amsfonts}
% \end{verbatim}
% followed by, e.g., \verb+\mathbb{R}+, \verb+\mathbb{N}+, or \verb+\mathbb{C}+
% for $\mathbb{R}$, $\mathbb{N}$ or $\mathbb{C}$.  You can also use the following
% workaround for reals, natural and complex:
% \begin{verbatim}
%    \newcommand{\RR}{I\!\!R} %real numbers
%    \newcommand{\Nat}{I\!\!N} %natural numbers
%    \newcommand{\CC}{I\!\!\!\!C} %complex numbers
% \end{verbatim}
% Note that \verb+amsfonts+ is automatically loaded by the \verb+amssymb+ package.


% \end{itemize}


% If your file contains type 3 fonts or non embedded TrueType fonts, we will ask
% you to fix it.


% \subsection{Margins in \LaTeX{}}


% Most of the margin problems come from figures positioned by hand using
% \verb+\special+ or other commands. We suggest using the command
% \verb+\includegraphics+ from the \verb+graphicx+ package. Always specify the
% figure width as a multiple of the line width as in the example below:
% \begin{verbatim}
%    \usepackage[pdftex]{graphicx} ...
%    \includegraphics[width=0.8\linewidth]{myfile.pdf}
% \end{verbatim}
% See Section 4.4 in the graphics bundle documentation
% (\url{http://mirrors.ctan.org/macros/latex/required/graphics/grfguide.pdf})


% A number of width problems arise when \LaTeX{} cannot properly hyphenate a
% line. Please give LaTeX hyphenation hints using the \verb+\-+ command when
% necessary.

% \begin{ack}
% Use unnumbered first level headings for the acknowledgments. All acknowledgments
% go at the end of the paper before the list of references. Moreover, you are required to declare
% funding (financial activities supporting the submitted work) and competing interests (related financial activities outside the submitted work).
% More information about this disclosure can be found at: \url{https://neurips.cc/Conferences/2024/PaperInformation/FundingDisclosure}.


% Do {\bf not} include this section in the anonymized submission, only in the final paper. You can use the \texttt{ack} environment provided in the style file to automatically hide this section in the anonymized submission.
% \end{ack}

% \section*{References}


% References follow the acknowledgments in the camera-ready paper. Use unnumbered first-level heading for
% the references. Any choice of citation style is acceptable as long as you are
% consistent. It is permissible to reduce the font size to \verb+small+ (9 point)
% when listing the references.
% Note that the Reference section does not count towards the page limit.
% \medskip


% {
% \small


% [1] Alexander, J.A.\ \& Mozer, M.C.\ (1995) Template-based algorithms for
% connectionist rule extraction. In G.\ Tesauro, D.S.\ Touretzky and T.K.\ Leen
% (eds.), {\it Advances in Neural Information Processing Systems 7},
% pp.\ 609--616. Cambridge, MA: MIT Press.


% [2] Bower, J.M.\ \& Beeman, D.\ (1995) {\it The Book of GENESIS: Exploring
%   Realistic Neural Models with the GEneral NEural SImulation System.}  New York:
% TELOS/Springer--Verlag.


% [3] Hasselmo, M.E., Schnell, E.\ \& Barkai, E.\ (1995) Dynamics of learning and
% recall at excitatory recurrent synapses and cholinergic modulation in rat
% hippocampal region CA3. {\it Journal of Neuroscience} {\bf 15}(7):5249-5262.
% }


% %%%%%%%%%%%%%%%%%%%%%%%%%%%%%%%%%%%%%%%%%%%%%%%%%%%%%%%%%%%%

% \appendix

% \section{Appendix / supplemental material}


% Optionally include supplemental material (complete proofs, additional experiments and plots) in appendix.
% All such materials \textbf{SHOULD be included in the main submission.}

% %%%%%%%%%%%%%%%%%%%%%%%%%%%%%%%%%%%%%%%%%%%%%%%%%%%%%%%%%%%%

% \newpage
% \section*{NeurIPS Paper Checklist}

% %%% BEGIN INSTRUCTIONS %%%
% The checklist is designed to encourage best practices for responsible machine learning research, addressing issues of reproducibility, transparency, research ethics, and societal impact. Do not remove the checklist: {\bf The papers not including the checklist will be desk rejected.} The checklist should follow the references and follow the (optional) supplemental material.  The checklist does NOT count towards the page
% limit. 

% Please read the checklist guidelines carefully for information on how to answer these questions. For each question in the checklist:
% \begin{itemize}
%     \item You should answer \answerYes{}, \answerNo{}, or \answerNA{}.
%     \item \answerNA{} means either that the question is Not Applicable for that particular paper or the relevant information is Not Available.
%     \item Please provide a short (1–2 sentence) justification right after your answer (even for NA). 
%    % \item {\bf The papers not including the checklist will be desk rejected.}
% \end{itemize}

% {\bf The checklist answers are an integral part of your paper submission.} They are visible to the reviewers, area chairs, senior area chairs, and ethics reviewers. You will be asked to also include it (after eventual revisions) with the final version of your paper, and its final version will be published with the paper.

% The reviewers of your paper will be asked to use the checklist as one of the factors in their evaluation. While "\answerYes{}" is generally preferable to "\answerNo{}", it is perfectly acceptable to answer "\answerNo{}" provided a proper justification is given (e.g., "error bars are not reported because it would be too computationally expensive" or "we were unable to find the license for the dataset we used"). In general, answering "\answerNo{}" or "\answerNA{}" is not grounds for rejection. While the questions are phrased in a binary way, we acknowledge that the true answer is often more nuanced, so please just use your best judgment and write a justification to elaborate. All supporting evidence can appear either in the main paper or the supplemental material, provided in appendix. If you answer \answerYes{} to a question, in the justification please point to the section(s) where related material for the question can be found.

% IMPORTANT, please:
% \begin{itemize}
%     \item {\bf Delete this instruction block, but keep the section heading ``NeurIPS paper checklist"},
%     \item  {\bf Keep the checklist subsection headings, questions/answers and guidelines below.}
%     \item {\bf Do not modify the questions and only use the provided macros for your answers}.
% \end{itemize} 
 

% %%% END INSTRUCTIONS %%%


% \begin{enumerate}

% \item {\bf Claims}
%     \item[] Question: Do the main claims made in the abstract and introduction accurately reflect the paper's contributions and scope?
%     \item[] Answer: \answerTODO{} % Replace by \answerYes{}, \answerNo{}, or \answerNA{}.
%     \item[] Justification: \justificationTODO{}
%     \item[] Guidelines:
%     \begin{itemize}
%         \item The answer NA means that the abstract and introduction do not include the claims made in the paper.
%         \item The abstract and/or introduction should clearly state the claims made, including the contributions made in the paper and important assumptions and limitations. A No or NA answer to this question will not be perceived well by the reviewers. 
%         \item The claims made should match theoretical and experimental results, and reflect how much the results can be expected to generalize to other settings. 
%         \item It is fine to include aspirational goals as motivation as long as it is clear that these goals are not attained by the paper. 
%     \end{itemize}

% \item {\bf Limitations}
%     \item[] Question: Does the paper discuss the limitations of the work performed by the authors?
%     \item[] Answer: \answerTODO{} % Replace by \answerYes{}, \answerNo{}, or \answerNA{}.
%     \item[] Justification: \justificationTODO{}
%     \item[] Guidelines:
%     \begin{itemize}
%         \item The answer NA means that the paper has no limitation while the answer No means that the paper has limitations, but those are not discussed in the paper. 
%         \item The authors are encouraged to create a separate "Limitations" section in their paper.
%         \item The paper should point out any strong assumptions and how robust the results are to violations of these assumptions (e.g., independence assumptions, noiseless settings, model well-specification, asymptotic approximations only holding locally). The authors should reflect on how these assumptions might be violated in practice and what the implications would be.
%         \item The authors should reflect on the scope of the claims made, e.g., if the approach was only tested on a few datasets or with a few runs. In general, empirical results often depend on implicit assumptions, which should be articulated.
%         \item The authors should reflect on the factors that influence the performance of the approach. For example, a facial recognition algorithm may perform poorly when image resolution is low or images are taken in low lighting. Or a speech-to-text system might not be used reliably to provide closed captions for online lectures because it fails to handle technical jargon.
%         \item The authors should discuss the computational efficiency of the proposed algorithms and how they scale with dataset size.
%         \item If applicable, the authors should discuss possible limitations of their approach to address problems of privacy and fairness.
%         \item While the authors might fear that complete honesty about limitations might be used by reviewers as grounds for rejection, a worse outcome might be that reviewers discover limitations that aren't acknowledged in the paper. The authors should use their best judgment and recognize that individual actions in favor of transparency play an important role in developing norms that preserve the integrity of the community. Reviewers will be specifically instructed to not penalize honesty concerning limitations.
%     \end{itemize}

% \item {\bf Theory Assumptions and Proofs}
%     \item[] Question: For each theoretical result, does the paper provide the full set of assumptions and a complete (and correct) proof?
%     \item[] Answer: \answerTODO{} % Replace by \answerYes{}, \answerNo{}, or \answerNA{}.
%     \item[] Justification: \justificationTODO{}
%     \item[] Guidelines:
%     \begin{itemize}
%         \item The answer NA means that the paper does not include theoretical results. 
%         \item All the theorems, formulas, and proofs in the paper should be numbered and cross-referenced.
%         \item All assumptions should be clearly stated or referenced in the statement of any theorems.
%         \item The proofs can either appear in the main paper or the supplemental material, but if they appear in the supplemental material, the authors are encouraged to provide a short proof sketch to provide intuition. 
%         \item Inversely, any informal proof provided in the core of the paper should be complemented by formal proofs provided in appendix or supplemental material.
%         \item Theorems and Lemmas that the proof relies upon should be properly referenced. 
%     \end{itemize}

%     \item {\bf Experimental Result Reproducibility}
%     \item[] Question: Does the paper fully disclose all the information needed to reproduce the main experimental results of the paper to the extent that it affects the main claims and/or conclusions of the paper (regardless of whether the code and data are provided or not)?
%     \item[] Answer: \answerTODO{} % Replace by \answerYes{}, \answerNo{}, or \answerNA{}.
%     \item[] Justification: \justificationTODO{}
%     \item[] Guidelines:
%     \begin{itemize}
%         \item The answer NA means that the paper does not include experiments.
%         \item If the paper includes experiments, a No answer to this question will not be perceived well by the reviewers: Making the paper reproducible is important, regardless of whether the code and data are provided or not.
%         \item If the contribution is a dataset and/or model, the authors should describe the steps taken to make their results reproducible or verifiable. 
%         \item Depending on the contribution, reproducibility can be accomplished in various ways. For example, if the contribution is a novel architecture, describing the architecture fully might suffice, or if the contribution is a specific model and empirical evaluation, it may be necessary to either make it possible for others to replicate the model with the same dataset, or provide access to the model. In general. releasing code and data is often one good way to accomplish this, but reproducibility can also be provided via detailed instructions for how to replicate the results, access to a hosted model (e.g., in the case of a large language model), releasing of a model checkpoint, or other means that are appropriate to the research performed.
%         \item While NeurIPS does not require releasing code, the conference does require all submissions to provide some reasonable avenue for reproducibility, which may depend on the nature of the contribution. For example
%         \begin{enumerate}
%             \item If the contribution is primarily a new algorithm, the paper should make it clear how to reproduce that algorithm.
%             \item If the contribution is primarily a new model architecture, the paper should describe the architecture clearly and fully.
%             \item If the contribution is a new model (e.g., a large language model), then there should either be a way to access this model for reproducing the results or a way to reproduce the model (e.g., with an open-source dataset or instructions for how to construct the dataset).
%             \item We recognize that reproducibility may be tricky in some cases, in which case authors are welcome to describe the particular way they provide for reproducibility. In the case of closed-source models, it may be that access to the model is limited in some way (e.g., to registered users), but it should be possible for other researchers to have some path to reproducing or verifying the results.
%         \end{enumerate}
%     \end{itemize}


% \item {\bf Open access to data and code}
%     \item[] Question: Does the paper provide open access to the data and code, with sufficient instructions to faithfully reproduce the main experimental results, as described in supplemental material?
%     \item[] Answer: \answerTODO{} % Replace by \answerYes{}, \answerNo{}, or \answerNA{}.
%     \item[] Justification: \justificationTODO{}
%     \item[] Guidelines:
%     \begin{itemize}
%         \item The answer NA means that paper does not include experiments requiring code.
%         \item Please see the NeurIPS code and data submission guidelines (\url{https://nips.cc/public/guides/CodeSubmissionPolicy}) for more details.
%         \item While we encourage the release of code and data, we understand that this might not be possible, so “No” is an acceptable answer. Papers cannot be rejected simply for not including code, unless this is central to the contribution (e.g., for a new open-source benchmark).
%         \item The instructions should contain the exact command and environment needed to run to reproduce the results. See the NeurIPS code and data submission guidelines (\url{https://nips.cc/public/guides/CodeSubmissionPolicy}) for more details.
%         \item The authors should provide instructions on data access and preparation, including how to access the raw data, preprocessed data, intermediate data, and generated data, etc.
%         \item The authors should provide scripts to reproduce all experimental results for the new proposed method and baselines. If only a subset of experiments are reproducible, they should state which ones are omitted from the script and why.
%         \item At submission time, to preserve anonymity, the authors should release anonymized versions (if applicable).
%         \item Providing as much information as possible in supplemental material (appended to the paper) is recommended, but including URLs to data and code is permitted.
%     \end{itemize}


% \item {\bf Experimental Setting/Details}
%     \item[] Question: Does the paper specify all the training and test details (e.g., data splits, hyperparameters, how they were chosen, type of optimizer, etc.) necessary to understand the results?
%     \item[] Answer: \answerTODO{} % Replace by \answerYes{}, \answerNo{}, or \answerNA{}.
%     \item[] Justification: \justificationTODO{}
%     \item[] Guidelines:
%     \begin{itemize}
%         \item The answer NA means that the paper does not include experiments.
%         \item The experimental setting should be presented in the core of the paper to a level of detail that is necessary to appreciate the results and make sense of them.
%         \item The full details can be provided either with the code, in appendix, or as supplemental material.
%     \end{itemize}

% \item {\bf Experiment Statistical Significance}
%     \item[] Question: Does the paper report error bars suitably and correctly defined or other appropriate information about the statistical significance of the experiments?
%     \item[] Answer: \answerTODO{} % Replace by \answerYes{}, \answerNo{}, or \answerNA{}.
%     \item[] Justification: \justificationTODO{}
%     \item[] Guidelines:
%     \begin{itemize}
%         \item The answer NA means that the paper does not include experiments.
%         \item The authors should answer "Yes" if the results are accompanied by error bars, confidence intervals, or statistical significance tests, at least for the experiments that support the main claims of the paper.
%         \item The factors of variability that the error bars are capturing should be clearly stated (for example, train/test split, initialization, random drawing of some parameter, or overall run with given experimental conditions).
%         \item The method for calculating the error bars should be explained (closed form formula, call to a library function, bootstrap, etc.)
%         \item The assumptions made should be given (e.g., Normally distributed errors).
%         \item It should be clear whether the error bar is the standard deviation or the standard error of the mean.
%         \item It is OK to report 1-sigma error bars, but one should state it. The authors should preferably report a 2-sigma error bar than state that they have a 96\% CI, if the hypothesis of Normality of errors is not verified.
%         \item For asymmetric distributions, the authors should be careful not to show in tables or figures symmetric error bars that would yield results that are out of range (e.g. negative error rates).
%         \item If error bars are reported in tables or plots, The authors should explain in the text how they were calculated and reference the corresponding figures or tables in the text.
%     \end{itemize}

% \item {\bf Experiments Compute Resources}
%     \item[] Question: For each experiment, does the paper provide sufficient information on the computer resources (type of compute workers, memory, time of execution) needed to reproduce the experiments?
%     \item[] Answer: \answerTODO{} % Replace by \answerYes{}, \answerNo{}, or \answerNA{}.
%     \item[] Justification: \justificationTODO{}
%     \item[] Guidelines:
%     \begin{itemize}
%         \item The answer NA means that the paper does not include experiments.
%         \item The paper should indicate the type of compute workers CPU or GPU, internal cluster, or cloud provider, including relevant memory and storage.
%         \item The paper should provide the amount of compute required for each of the individual experimental runs as well as estimate the total compute. 
%         \item The paper should disclose whether the full research project required more compute than the experiments reported in the paper (e.g., preliminary or failed experiments that didn't make it into the paper). 
%     \end{itemize}
    
% \item {\bf Code Of Ethics}
%     \item[] Question: Does the research conducted in the paper conform, in every respect, with the NeurIPS Code of Ethics \url{https://neurips.cc/public/EthicsGuidelines}?
%     \item[] Answer: \answerTODO{} % Replace by \answerYes{}, \answerNo{}, or \answerNA{}.
%     \item[] Justification: \justificationTODO{}
%     \item[] Guidelines:
%     \begin{itemize}
%         \item The answer NA means that the authors have not reviewed the NeurIPS Code of Ethics.
%         \item If the authors answer No, they should explain the special circumstances that require a deviation from the Code of Ethics.
%         \item The authors should make sure to preserve anonymity (e.g., if there is a special consideration due to laws or regulations in their jurisdiction).
%     \end{itemize}


% \item {\bf Broader Impacts}
%     \item[] Question: Does the paper discuss both potential positive societal impacts and negative societal impacts of the work performed?
%     \item[] Answer: \answerTODO{} % Replace by \answerYes{}, \answerNo{}, or \answerNA{}.
%     \item[] Justification: \justificationTODO{}
%     \item[] Guidelines:
%     \begin{itemize}
%         \item The answer NA means that there is no societal impact of the work performed.
%         \item If the authors answer NA or No, they should explain why their work has no societal impact or why the paper does not address societal impact.
%         \item Examples of negative societal impacts include potential malicious or unintended uses (e.g., disinformation, generating fake profiles, surveillance), fairness considerations (e.g., deployment of technologies that could make decisions that unfairly impact specific groups), privacy considerations, and security considerations.
%         \item The conference expects that many papers will be foundational research and not tied to particular applications, let alone deployments. However, if there is a direct path to any negative applications, the authors should point it out. For example, it is legitimate to point out that an improvement in the quality of generative models could be used to generate deepfakes for disinformation. On the other hand, it is not needed to point out that a generic algorithm for optimizing neural networks could enable people to train models that generate Deepfakes faster.
%         \item The authors should consider possible harms that could arise when the technology is being used as intended and functioning correctly, harms that could arise when the technology is being used as intended but gives incorrect results, and harms following from (intentional or unintentional) misuse of the technology.
%         \item If there are negative societal impacts, the authors could also discuss possible mitigation strategies (e.g., gated release of models, providing defenses in addition to attacks, mechanisms for monitoring misuse, mechanisms to monitor how a system learns from feedback over time, improving the efficiency and accessibility of ML).
%     \end{itemize}
    
% \item {\bf Safeguards}
%     \item[] Question: Does the paper describe safeguards that have been put in place for responsible release of data or models that have a high risk for misuse (e.g., pretrained language models, image generators, or scraped datasets)?
%     \item[] Answer: \answerTODO{} % Replace by \answerYes{}, \answerNo{}, or \answerNA{}.
%     \item[] Justification: \justificationTODO{}
%     \item[] Guidelines:
%     \begin{itemize}
%         \item The answer NA means that the paper poses no such risks.
%         \item Released models that have a high risk for misuse or dual-use should be released with necessary safeguards to allow for controlled use of the model, for example by requiring that users adhere to usage guidelines or restrictions to access the model or implementing safety filters. 
%         \item Datasets that have been scraped from the Internet could pose safety risks. The authors should describe how they avoided releasing unsafe images.
%         \item We recognize that providing effective safeguards is challenging, and many papers do not require this, but we encourage authors to take this into account and make a best faith effort.
%     \end{itemize}

% \item {\bf Licenses for existing assets}
%     \item[] Question: Are the creators or original owners of assets (e.g., code, data, models), used in the paper, properly credited and are the license and terms of use explicitly mentioned and properly respected?
%     \item[] Answer: \answerTODO{} % Replace by \answerYes{}, \answerNo{}, or \answerNA{}.
%     \item[] Justification: \justificationTODO{}
%     \item[] Guidelines:
%     \begin{itemize}
%         \item The answer NA means that the paper does not use existing assets.
%         \item The authors should cite the original paper that produced the code package or dataset.
%         \item The authors should state which version of the asset is used and, if possible, include a URL.
%         \item The name of the license (e.g., CC-BY 4.0) should be included for each asset.
%         \item For scraped data from a particular source (e.g., website), the copyright and terms of service of that source should be provided.
%         \item If assets are released, the license, copyright information, and terms of use in the package should be provided. For popular datasets, \url{paperswithcode.com/datasets} has curated licenses for some datasets. Their licensing guide can help determine the license of a dataset.
%         \item For existing datasets that are re-packaged, both the original license and the license of the derived asset (if it has changed) should be provided.
%         \item If this information is not available online, the authors are encouraged to reach out to the asset's creators.
%     \end{itemize}

% \item {\bf New Assets}
%     \item[] Question: Are new assets introduced in the paper well documented and is the documentation provided alongside the assets?
%     \item[] Answer: \answerTODO{} % Replace by \answerYes{}, \answerNo{}, or \answerNA{}.
%     \item[] Justification: \justificationTODO{}
%     \item[] Guidelines:
%     \begin{itemize}
%         \item The answer NA means that the paper does not release new assets.
%         \item Researchers should communicate the details of the dataset/code/model as part of their submissions via structured templates. This includes details about training, license, limitations, etc. 
%         \item The paper should discuss whether and how consent was obtained from people whose asset is used.
%         \item At submission time, remember to anonymize your assets (if applicable). You can either create an anonymized URL or include an anonymized zip file.
%     \end{itemize}

% \item {\bf Crowdsourcing and Research with Human Subjects}
%     \item[] Question: For crowdsourcing experiments and research with human subjects, does the paper include the full text of instructions given to participants and screenshots, if applicable, as well as details about compensation (if any)? 
%     \item[] Answer: \answerTODO{} % Replace by \answerYes{}, \answerNo{}, or \answerNA{}.
%     \item[] Justification: \justificationTODO{}
%     \item[] Guidelines:
%     \begin{itemize}
%         \item The answer NA means that the paper does not involve crowdsourcing nor research with human subjects.
%         \item Including this information in the supplemental material is fine, but if the main contribution of the paper involves human subjects, then as much detail as possible should be included in the main paper. 
%         \item According to the NeurIPS Code of Ethics, workers involved in data collection, curation, or other labor should be paid at least the minimum wage in the country of the data collector. 
%     \end{itemize}

% \item {\bf Institutional Review Board (IRB) Approvals or Equivalent for Research with Human Subjects}
%     \item[] Question: Does the paper describe potential risks incurred by study participants, whether such risks were disclosed to the subjects, and whether Institutional Review Board (IRB) approvals (or an equivalent approval/review based on the requirements of your country or institution) were obtained?
%     \item[] Answer: \answerTODO{} % Replace by \answerYes{}, \answerNo{}, or \answerNA{}.
%     \item[] Justification: \justificationTODO{}
%     \item[] Guidelines:
%     \begin{itemize}
%         \item The answer NA means that the paper does not involve crowdsourcing nor research with human subjects.
%         \item Depending on the country in which research is conducted, IRB approval (or equivalent) may be required for any human subjects research. If you obtained IRB approval, you should clearly state this in the paper. 
%         \item We recognize that the procedures for this may vary significantly between institutions and locations, and we expect authors to adhere to the NeurIPS Code of Ethics and the guidelines for their institution. 
%         \item For initial submissions, do not include any information that would break anonymity (if applicable), such as the institution conducting the review.
%     \end{itemize}

% \end{enumerate}


% \end{document}